% Options for packages loaded elsewhere
\PassOptionsToPackage{unicode}{hyperref}
\PassOptionsToPackage{hyphens}{url}
\PassOptionsToPackage{dvipsnames,svgnames,x11names}{xcolor}
%
\documentclass[
]{estat/estat}

\usepackage{amsmath,amssymb}
\usepackage{iftex}
\ifPDFTeX
  \usepackage[T1]{fontenc}
  \usepackage[utf8]{inputenc}
  \usepackage{textcomp} % provide euro and other symbols
\else % if luatex or xetex
  \usepackage{unicode-math}
  \defaultfontfeatures{Scale=MatchLowercase}
  \defaultfontfeatures[\rmfamily]{Ligatures=TeX,Scale=1}
\fi
\usepackage{lmodern}
\ifPDFTeX\else  
    % xetex/luatex font selection
    \setmainfont[]{Arial}
\fi
% Use upquote if available, for straight quotes in verbatim environments
\IfFileExists{upquote.sty}{\usepackage{upquote}}{}
\IfFileExists{microtype.sty}{% use microtype if available
  \usepackage[]{microtype}
  \UseMicrotypeSet[protrusion]{basicmath} % disable protrusion for tt fonts
}{}
\makeatletter
\@ifundefined{KOMAClassName}{% if non-KOMA class
  \IfFileExists{parskip.sty}{%
    \usepackage{parskip}
  }{% else
    \setlength{\parindent}{0pt}
    \setlength{\parskip}{6pt plus 2pt minus 1pt}}
}{% if KOMA class
  \KOMAoptions{parskip=half}}
\makeatother
\usepackage{xcolor}
\usepackage[left=3cm,right=2cm,top=3cm,bottom=2cm]{geometry}
\setlength{\emergencystretch}{3em} % prevent overfull lines
\setcounter{secnumdepth}{5}
% Make \paragraph and \subparagraph free-standing
\makeatletter
\ifx\paragraph\undefined\else
  \let\oldparagraph\paragraph
  \renewcommand{\paragraph}{
    \@ifstar
      \xxxParagraphStar
      \xxxParagraphNoStar
  }
  \newcommand{\xxxParagraphStar}[1]{\oldparagraph*{#1}\mbox{}}
  \newcommand{\xxxParagraphNoStar}[1]{\oldparagraph{#1}\mbox{}}
\fi
\ifx\subparagraph\undefined\else
  \let\oldsubparagraph\subparagraph
  \renewcommand{\subparagraph}{
    \@ifstar
      \xxxSubParagraphStar
      \xxxSubParagraphNoStar
  }
  \newcommand{\xxxSubParagraphStar}[1]{\oldsubparagraph*{#1}\mbox{}}
  \newcommand{\xxxSubParagraphNoStar}[1]{\oldsubparagraph{#1}\mbox{}}
\fi
\makeatother


\providecommand{\tightlist}{%
  \setlength{\itemsep}{0pt}\setlength{\parskip}{0pt}}\usepackage{longtable,booktabs,array}
\usepackage{calc} % for calculating minipage widths
% Correct order of tables after \paragraph or \subparagraph
\usepackage{etoolbox}
\makeatletter
\patchcmd\longtable{\par}{\if@noskipsec\mbox{}\fi\par}{}{}
\makeatother
% Allow footnotes in longtable head/foot
\IfFileExists{footnotehyper.sty}{\usepackage{footnotehyper}}{\usepackage{footnote}}
\makesavenoteenv{longtable}
\usepackage{graphicx}
\makeatletter
\newsavebox\pandoc@box
\newcommand*\pandocbounded[1]{% scales image to fit in text height/width
  \sbox\pandoc@box{#1}%
  \Gscale@div\@tempa{\textheight}{\dimexpr\ht\pandoc@box+\dp\pandoc@box\relax}%
  \Gscale@div\@tempb{\linewidth}{\wd\pandoc@box}%
  \ifdim\@tempb\p@<\@tempa\p@\let\@tempa\@tempb\fi% select the smaller of both
  \ifdim\@tempa\p@<\p@\scalebox{\@tempa}{\usebox\pandoc@box}%
  \else\usebox{\pandoc@box}%
  \fi%
}
% Set default figure placement to htbp
\def\fps@figure{htbp}
\makeatother

\authors{%
    Estatiano 1 \\
    Estatiano 2\\
    Estatiano 3\\
}

% escreva o nome do cliente aqui
% se for mais de um separe por \\
\client{%
    ESTAT
}
% Baixando pacotes
\RequirePackage{fancyhdr}
\RequirePackage{graphicx}

\setlength\headheight{28pt}  

\setlength{\parindent}{15pt} % Adiciona indentação nos parágrafos
\setlength{\parskip}{0pt} % Adiciona 0 espaço entro os parágrafos

\let\oldsection\section
\renewcommand\section{\clearpage\oldsection}
\makeatletter
\@ifpackageloaded{float}{}{\usepackage{float}}
\floatstyle{plain}
\@ifundefined{c@chapter}{\newfloat{quadro}{h}{loquad}}{\newfloat{quadro}{h}{loquad}[chapter]}
\floatname{quadro}{Quadro}
\floatstyle{plaintop}
\restylefloat{quadro}
\newcommand*\listofquadros{\listof{quadro}{List of Testes}}
\makeatother
\makeatletter
\@ifpackageloaded{caption}{}{\usepackage{caption}}
\AtBeginDocument{%
\ifdefined\contentsname
  \renewcommand*\contentsname{Índice}
\else
  \newcommand\contentsname{Índice}
\fi
\ifdefined\listfigurename
  \renewcommand*\listfigurename{Lista de Figuras}
\else
  \newcommand\listfigurename{Lista de Figuras}
\fi
\ifdefined\listtablename
  \renewcommand*\listtablename{Lista de Tabelas}
\else
  \newcommand\listtablename{Lista de Tabelas}
\fi
\ifdefined\figurename
  \renewcommand*\figurename{Figura}
\else
  \newcommand\figurename{Figura}
\fi
\ifdefined\tablename
  \renewcommand*\tablename{Tabela}
\else
  \newcommand\tablename{Tabela}
\fi
}
\@ifpackageloaded{float}{}{\usepackage{float}}
\floatstyle{ruled}
\@ifundefined{c@chapter}{\newfloat{codelisting}{h}{lop}}{\newfloat{codelisting}{h}{lop}[chapter]}
\floatname{codelisting}{Listagem}
\newcommand*\listoflistings{\listof{codelisting}{Lista de Listagens}}
\captionsetup{labelsep=colon}
\makeatother
\makeatletter
\makeatother
\makeatletter
\@ifpackageloaded{caption}{}{\usepackage{caption}}
\@ifpackageloaded{subcaption}{}{\usepackage{subcaption}}
\makeatother

\ifLuaTeX
\usepackage[bidi=basic]{babel}
\else
\usepackage[bidi=default]{babel}
\fi
\babelprovide[main,import]{portuguese}
\ifPDFTeX
\else
\babelfont{rm}[]{Arial}
\fi
% get rid of language-specific shorthands (see #6817):
\let\LanguageShortHands\languageshorthands
\def\languageshorthands#1{}
\usepackage{bookmark}

\IfFileExists{xurl.sty}{\usepackage{xurl}}{} % add URL line breaks if available
\urlstyle{same} % disable monospaced font for URLs
\hypersetup{
  pdflang={pt},
  colorlinks=true,
  linkcolor={black},
  filecolor={black},
  citecolor={black},
  urlcolor={black},
  pdfcreator={LaTeX via pandoc}}


\author{}
\date{}

\begin{document}

% Limpando tudo
\fancyhf{} 

% Ajustes do header
\fancyhead[L]{} % limpando o lado esquerdo
\fancyhead[R]{\includegraphics[width=0.20\textwidth]{estat/imagens/estat.png}} % adicionando logo no canto direito
\renewcommand{\headrulewidth}{0pt}   % sem linha embaixo da logo

% Ajustes de fim de página
\fancyfoot[R]{\textcolor{white}{\thepage}} % Número em branco no canto direito

% Aplicando o estilo que acabamos de criar
\pagestyle{fancy} 


\labelformat{quadro}{\textbf{#1}}

\renewcommand*\contentsname{Sumário}
{
\hypersetup{linkcolor=}
\setcounter{tocdepth}{3}
\tableofcontents
}

\section{Objetivo}\label{objetivo}

Esse template foi criado para o alocado conseguir observar como ficaria
sua análise o arquivo principal. É daqui que o gerente de projetos irá
copiar a análise e inserir no documento principal que gerará o relatório
estatístico.

\section{Análises}\label{anuxe1lises}

\subsection{Análise dos perfis das idades dos clientes para cada loja da
cidade de Âmbar
Seco.}\label{anuxe1lise-dos-perfis-das-idades-dos-clientes-para-cada-loja-da-cidade-de-uxe2mbar-seco.}

Com o objetivo de entender o perfil da idade dos clientes de Âmbar Seco
por loja, foi realizada uma análise descritivas. Os dados foram obtidos
por meio da junção das tabelas ``infos\_vendas'', ``infos\_produtos'',
``infos\_clientes'', ``infos\_lojas'', ``infos\_cidades''.
Adicionalmente, o banco de dados foi filtrado para incluir apenas
clientes de Âmbar Seco e ajustado para considerar somente clientes
únicos, garantindo que a análise do perfil etário não fosse enviesada
pelo volume de transações. Além disso, como se tratam de uma variável
quantitativa contínua (Idade) e uma variável qualitativa nominal (
Número da Loja), foi feito um diagrama de caixas bivariado (Figura 3),
em função do número da Loja, e uma tabela de medidas resumo (Tabela 2),
contendo: média, mediana, quartis, desvio-padrão.

\begin{figure}[H]

\caption{\label{fig-3}Boxplot da idade (anos) pelo número da loja de
Âmbar Seco}

\centering{

\pandocbounded{\includegraphics[keepaspectratio]{entrega_3_files/figure-pdf/fig-3-1.pdf}}

}

\end{figure}%

O diagrama foi construido, de modo que, está ordenado pela ordem
crescente da média, que é representada pelo losango branco. A Loja 9
possui a menor média de idade das lojas, 33,67, e o segundo menor
mínimo, 16 anos, além disso, metade das pessoas que vão a essa Loja têm
entre 24 e 39 anos, e possui o maior limite superior, 80 anos, além de
vários outliers. Já a Loja 8, possui a segunda menor média entre as
lojas, 34,2, e o menor mínimo, 15 anos, metade de seus consumidores tem
entre 26 e 38 anos e apresenta valores átipicos maiores do que o seu
limite superior. A Loja 2 não possui valores discrepantes, e 50\% dos
seus clientes tem entre 31 e 40 anos, seu limite infeirior é 25 anos, o
que já é maior que a média da loja 9, e seu máximo é 44 anos, diferente
das Lojas 8 e 9 que possuem máximos maiores e outliers, isso indica que
a idade é mais próxima entre os clientes que a frequentam, o que é
confirmado pelo seu desvio-padrão baixo, de 5 anos, o menor desvio
padrão, além de ter a menor variância, 31,06. Por fim, a Loja 6, possui
a maior média, 40,35 anos de idade, e assim como a loja 2, não possui
outliers e sua assimetria é muito menor que nas Lojas 8 e 9, e metade de
seus clientes tem entre 35 e 46 anos. Portanto, as Lojas 2 e 6, com
menores desvios padrão, 5,47 e 6,03, respectivamente, demonstram um
público alvo etário mais homogêneo e definido, concentrado entre
aproximadamente 30 e 45 anos. Em contraste, as Lojas 8 e 9, com altos
desvios-padrão, 12,7 e 13,31, indicam uma grande heterogeneidade etária
devido à presença de valores atípicos e maior dispersão.

\begin{quadro}[H]

\caption{\label{quad-resumo}Medidas resumo da}

\centering{

\begin{quadro}[H]
    \setlength{ \tabcolsep}{9pt}
    \renewcommand{  \arraystretch}{1.20}
    \caption{Medidas resumo da(o) [nome da variável]}
    \centering
    \begin{adjustbox}{max width=\textwidth}
    \begin{tabular} { | l |
            S[table-format = 2.2]
            S[table-format = 2.2]
            S[table-format = 3.2]
            S[table-format = 2.2]
            |}
    \hline
        \textbf{Estatística} & \textbf{2} & \textbf{6} & \textbf{8} & \textbf{9} \\
        \hline
        Média & 34,92 & 40,35 & 34,2 & 33,67 \\
        Desvio Padrão & 5,57 & 6,03 & 12,7 & 13,31 \\
        Variância & 31,06 & 36,39 & 161,23 & 177,18 \\
        Mínimo & 25 & 30 & 15 & 16 \\
        1º Quartil & 31 & 35 & 26 & 24 \\
        Mediana & 34 & 41 & 32,5 & 30,5 \\
        3º Quartil & 40 & 46 & 38 & 39 \\
        Máximo & 44 & 49 & 77 & 80 \\
    \hline
    \end{tabular}
    
    \end{adjustbox}
\end{quadro}

}

\end{quadro}%

\begin{table}[H]

\caption{\label{tbl-modalidades}}

\centering{

\begin{quadro}[H]
    \setlength{ \tabcolsep}{9pt}
    \renewcommand{  \arraystretch}{1.20}
    \caption{Medidas resumo da(o) [nome da variável]}
    \centering
    \begin{adjustbox}{max width=\textwidth}
    \begin{tabular} { | l |
            S[table-format = 2.2]
            S[table-format = 2.2]
            S[table-format = 3.2]
            S[table-format = 2.2]
            |}
    \hline
        \textbf{Estatística} & \textbf{2} & \textbf{6} & \textbf{8} & \textbf{9} \\
        \hline
        Média & 34,92 & 40,35 & 34,2 & 33,67 \\
        Desvio Padrão & 5,57 & 6,03 & 12,7 & 13,31 \\
        Variância & 31,06 & 36,39 & 161,23 & 177,18 \\
        Mínimo & 25 & 30 & 15 & 16 \\
        1º Quartil & 31 & 35 & 26 & 24 \\
        Mediana & 34 & 41 & 32,5 & 30,5 \\
        3º Quartil & 40 & 46 & 38 & 39 \\
        Máximo & 44 & 49 & 77 & 80 \\
    \hline
    \end{tabular}
    
    \end{adjustbox}
\end{quadro}

}

\end{table}%

\(\ref{fig-}\)

{[}@tbl-{]}

{[}**Quadro** @quad-{]}




\end{document}
